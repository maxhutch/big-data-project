\documentclass{article}
\usepackage{amsmath, amsthm, amssymb}
\usepackage{tikz}
\usepackage{subfig}
\usepackage{verbatim}
\usepackage[all]{xy}
\usetikzlibrary{positioning}

\newcommand{\set}[1] {{\left\{#1\right\}}}


\begin{document}


\title{Data Intensive Computing Project Plan}
\author{Max Hutchinson}
\date{\today}
\maketitle

This project aims to provide insights to the question: ``Are data intensive computing systems effective tools for computational scientists?''  For the purposes of this project, a \textit{computational scientist} is represented by one who is trying to solve a particular problem: the enumeration of random tilings (i.e. me).  \textit{Efficiency} will be considered on a number of fronts: time efficiency, memory efficiency, disk efficiency, network efficiency, and programmer efficiency.  Each of these evaluations will consider three scales: development (workstation), production (cluster), and heroic (largest possible system).

The problem of random tiling enumeration is important to the understanding of quasicrystals~\cite{Shechtman1984,Levine1984}.  From an algorithms point of view, it is a breadth first operation~\cite{Dest01} on a DAG with a convenient layering property: all paths between two nodes have the same length.  This operation is difficult for three reasons: 1) the DAG grows super-exponentially with the problem size, 2) the DAG is only weakly local, and 3) the memory requirements for the data on the DAG edges is dynamic.  The DAGs that arise in random tiling enumerations have sizes, edge sparsities, data densities, and concurrent widths that are tunable and known a-priori. 

The experiment will consist of implementing a random tiling enumerator on a variety of data intensive computing systems.  The first, which is already being used in production, is based on simple C with POSIX threads (Pthreads)~\cite{Mueller1993}.  The second will be in Hadoop~\cite{Bialecki2005,Borthakur2008}, a popular implementation of MapReduce.  The third will be in GraphLab~\cite{Low2012} and, capabilities permitting, GraphChi~\cite{Kyrola2012}, which use a vertex computation model.

These three data intensive computing systems represent very different philosophies for efficiency.  For example, MapReduce is highly restrictive but highly automatic, while Pthreads is highly expressive but handles almost nothing automatically.  GraphLab tries to carve out a middle ground, being more expressive than MapReduce and more automatic than Pthreads.  Hypothesized results can be seen in Table~\ref{tbl:hyp}.

Observing the effects of these design philosophies on this particular scientific workload will go a long way towards evaluating the effectiveness of data intensive computing systems, such as MapReduce and GraphLab, on problems which have been historically handled explicitly in a minimalistic environment such as Pthreads.  Analysis will focus on systems issues, ``How well does the framework take advantage of the hardware in the context of the task?'', over programmatic issues, ``How well does the framework allow for the expression of the problem?'', though both will be considered. 

Although focused on scientific workloads and audiences, the conclusions drawn from the direct comparison of three very different design philosophies should be relevant to the data intensive computing community at large.  In particular, this project will result in implementations of a very regular yet scalable problem on three popular data intensive computing systems.  They could, in a sense, be considered application benchmarks,thus serving a range of purposes from hardware comparisons to diagnostics.  The implementations used in this project will be made publicly available. 

\begin{table}
\centering
\begin{tabular}{l | l l l }
 & \multicolumn{3}{c}{Scale} \\
Category & Development & Production & Heroic \\
\hline
Time & Pthreads & GraphLab & Hadoop \\ 
Memory & Pthreads & Pthreads & Pthreads \\ 
Disk & GraphChi & GraphChi & Hadoop \\
Network & Pthreads & GraphLab & GraphLab \\
Programmer & Hadoop & Hadoop & Hadoop
\end{tabular}
\caption{Hypothesized winners for each efficiency category at each problem scale.}
\label{tbl:hyp}
\end{table}

\bibliographystyle{plain}
\bibliography{../../../writings/bib/library}

\end{document}
